
% Default to the notebook output style


% Inherit from the specified cell style.





    
\documentclass[11pt]{article}

    
    

    \usepackage[T1]{fontenc}
    % Nicer default font (+ math font) than Computer Modern for most use cases
    \usepackage{mathpazo}

    % Basic figure setup, for now with no caption control since it's done
    % automatically by Pandoc (which extracts ![](path) syntax from Markdown).
    \usepackage{graphicx}
    % We will generate all images so they have a width \maxwidth. This means
    % that they will get their normal width if they fit onto the page, but
    % are scaled down if they would overflow the margins.
    \makeatletter
    \def\maxwidth{\ifdim\Gin@nat@width>\linewidth\linewidth
    \else\Gin@nat@width\fi}
    \makeatother
    \let\Oldincludegraphics\includegraphics
    % Set max figure width to be 80% of text width, for now hardcoded.
    \renewcommand{\includegraphics}[1]{\Oldincludegraphics[width=.8\maxwidth]{#1}}
    % Ensure that by default, figures have no caption (until we provide a
    % proper Figure object with a Caption API and a way to capture that
    % in the conversion process - todo).
    \usepackage{caption}
    \DeclareCaptionLabelFormat{nolabel}{}
    \captionsetup{labelformat=nolabel}

    \usepackage{adjustbox} % Used to constrain images to a maximum size 
    \usepackage{xcolor} % Allow colors to be defined
    \usepackage{enumerate} % Needed for markdown enumerations to work
    \usepackage{geometry} % Used to adjust the document margins
    \usepackage{amsmath} % Equations
    \usepackage{amssymb} % Equations
    \usepackage{textcomp} % defines textquotesingle
    % Hack from http://tex.stackexchange.com/a/47451/13684:
    \AtBeginDocument{%
        \def\PYZsq{\textquotesingle}% Upright quotes in Pygmentized code
    }
    \usepackage{upquote} % Upright quotes for verbatim code
    \usepackage{eurosym} % defines \euro
    \usepackage[mathletters]{ucs} % Extended unicode (utf-8) support
    \usepackage[utf8x]{inputenc} % Allow utf-8 characters in the tex document
    \usepackage{fancyvrb} % verbatim replacement that allows latex
    \usepackage{grffile} % extends the file name processing of package graphics 
                         % to support a larger range 
    % The hyperref package gives us a pdf with properly built
    % internal navigation ('pdf bookmarks' for the table of contents,
    % internal cross-reference links, web links for URLs, etc.)
    \usepackage{hyperref}
    \usepackage{longtable} % longtable support required by pandoc >1.10
    \usepackage{booktabs}  % table support for pandoc > 1.12.2
    \usepackage[inline]{enumitem} % IRkernel/repr support (it uses the enumerate* environment)
    \usepackage[normalem]{ulem} % ulem is needed to support strikethroughs (\sout)
                                % normalem makes italics be italics, not underlines
    
    \usepackage{listings} % Used to define pretty listings for code sections [jfb]
    \usepackage{float}   


    
    
    % Colors for the hyperref package
    \definecolor{urlcolor}{rgb}{0,.145,.698}
    \definecolor{linkcolor}{rgb}{.71,0.21,0.01}
    \definecolor{citecolor}{rgb}{.12,.54,.11}

    % ANSI colors
    \definecolor{ansi-black}{HTML}{3E424D}
    \definecolor{ansi-black-intense}{HTML}{282C36}
    \definecolor{ansi-red}{HTML}{E75C58}
    \definecolor{ansi-red-intense}{HTML}{B22B31}
    \definecolor{ansi-green}{HTML}{00A250}
    \definecolor{ansi-green-intense}{HTML}{007427}
    \definecolor{ansi-yellow}{HTML}{DDB62B}
    \definecolor{ansi-yellow-intense}{HTML}{B27D12}
    \definecolor{ansi-blue}{HTML}{208FFB}
    \definecolor{ansi-blue-intense}{HTML}{0065CA}
    \definecolor{ansi-magenta}{HTML}{D160C4}
    \definecolor{ansi-magenta-intense}{HTML}{A03196}
    \definecolor{ansi-cyan}{HTML}{60C6C8}
    \definecolor{ansi-cyan-intense}{HTML}{258F8F}
    \definecolor{ansi-white}{HTML}{C5C1B4}
    \definecolor{ansi-white-intense}{HTML}{A1A6B2}

    % commands and environments needed by pandoc snippets
    % extracted from the output of `pandoc -s`
    \providecommand{\tightlist}{%
      \setlength{\itemsep}{0pt}\setlength{\parskip}{0pt}}
    \DefineVerbatimEnvironment{Highlighting}{Verbatim}{commandchars=\\\{\}}
    % Add ',fontsize=\small' for more characters per line
    \newenvironment{Shaded}{}{}
    \newcommand{\KeywordTok}[1]{\textcolor[rgb]{0.00,0.44,0.13}{\textbf{{#1}}}}
    \newcommand{\DataTypeTok}[1]{\textcolor[rgb]{0.56,0.13,0.00}{{#1}}}
    \newcommand{\DecValTok}[1]{\textcolor[rgb]{0.25,0.63,0.44}{{#1}}}
    \newcommand{\BaseNTok}[1]{\textcolor[rgb]{0.25,0.63,0.44}{{#1}}}
    \newcommand{\FloatTok}[1]{\textcolor[rgb]{0.25,0.63,0.44}{{#1}}}
    \newcommand{\CharTok}[1]{\textcolor[rgb]{0.25,0.44,0.63}{{#1}}}
    \newcommand{\StringTok}[1]{\textcolor[rgb]{0.25,0.44,0.63}{{#1}}}
    \newcommand{\CommentTok}[1]{\textcolor[rgb]{0.38,0.63,0.69}{\textit{{#1}}}}
    \newcommand{\OtherTok}[1]{\textcolor[rgb]{0.00,0.44,0.13}{{#1}}}
    \newcommand{\AlertTok}[1]{\textcolor[rgb]{1.00,0.00,0.00}{\textbf{{#1}}}}
    \newcommand{\FunctionTok}[1]{\textcolor[rgb]{0.02,0.16,0.49}{{#1}}}
    \newcommand{\RegionMarkerTok}[1]{{#1}}
    \newcommand{\ErrorTok}[1]{\textcolor[rgb]{1.00,0.00,0.00}{\textbf{{#1}}}}
    \newcommand{\NormalTok}[1]{{#1}}
    
    % Additional commands for more recent versions of Pandoc
    \newcommand{\ConstantTok}[1]{\textcolor[rgb]{0.53,0.00,0.00}{{#1}}}
    \newcommand{\SpecialCharTok}[1]{\textcolor[rgb]{0.25,0.44,0.63}{{#1}}}
    \newcommand{\VerbatimStringTok}[1]{\textcolor[rgb]{0.25,0.44,0.63}{{#1}}}
    \newcommand{\SpecialStringTok}[1]{\textcolor[rgb]{0.73,0.40,0.53}{{#1}}}
    \newcommand{\ImportTok}[1]{{#1}}
    \newcommand{\DocumentationTok}[1]{\textcolor[rgb]{0.73,0.13,0.13}{\textit{{#1}}}}
    \newcommand{\AnnotationTok}[1]{\textcolor[rgb]{0.38,0.63,0.69}{\textbf{\textit{{#1}}}}}
    \newcommand{\CommentVarTok}[1]{\textcolor[rgb]{0.38,0.63,0.69}{\textbf{\textit{{#1}}}}}
    \newcommand{\VariableTok}[1]{\textcolor[rgb]{0.10,0.09,0.49}{{#1}}}
    \newcommand{\ControlFlowTok}[1]{\textcolor[rgb]{0.00,0.44,0.13}{\textbf{{#1}}}}
    \newcommand{\OperatorTok}[1]{\textcolor[rgb]{0.40,0.40,0.40}{{#1}}}
    \newcommand{\BuiltInTok}[1]{{#1}}
    \newcommand{\ExtensionTok}[1]{{#1}}
    \newcommand{\PreprocessorTok}[1]{\textcolor[rgb]{0.74,0.48,0.00}{{#1}}}
    \newcommand{\AttributeTok}[1]{\textcolor[rgb]{0.49,0.56,0.16}{{#1}}}
    \newcommand{\InformationTok}[1]{\textcolor[rgb]{0.38,0.63,0.69}{\textbf{\textit{{#1}}}}}
    \newcommand{\WarningTok}[1]{\textcolor[rgb]{0.38,0.63,0.69}{\textbf{\textit{{#1}}}}}
    
    
    % Define a nice break command that doesn't care if a line doesn't already
    % exist.
    \def\br{\hspace*{\fill} \\* }
    % Math Jax compatability definitions
    \def\gt{>}
    \def\lt{<}
    % Document parameters
    
\title{ }

    
    
\author{J.-F. Bercher}

    

    % Pygments definitions
    
\makeatletter
\def\PY@reset{\let\PY@it=\relax \let\PY@bf=\relax%
    \let\PY@ul=\relax \let\PY@tc=\relax%
    \let\PY@bc=\relax \let\PY@ff=\relax}
\def\PY@tok#1{\csname PY@tok@#1\endcsname}
\def\PY@toks#1+{\ifx\relax#1\empty\else%
    \PY@tok{#1}\expandafter\PY@toks\fi}
\def\PY@do#1{\PY@bc{\PY@tc{\PY@ul{%
    \PY@it{\PY@bf{\PY@ff{#1}}}}}}}
\def\PY#1#2{\PY@reset\PY@toks#1+\relax+\PY@do{#2}}

\expandafter\def\csname PY@tok@w\endcsname{\def\PY@tc##1{\textcolor[rgb]{0.73,0.73,0.73}{##1}}}
\expandafter\def\csname PY@tok@c\endcsname{\let\PY@it=\textit\def\PY@tc##1{\textcolor[rgb]{0.25,0.50,0.50}{##1}}}
\expandafter\def\csname PY@tok@cp\endcsname{\def\PY@tc##1{\textcolor[rgb]{0.74,0.48,0.00}{##1}}}
\expandafter\def\csname PY@tok@k\endcsname{\let\PY@bf=\textbf\def\PY@tc##1{\textcolor[rgb]{0.00,0.50,0.00}{##1}}}
\expandafter\def\csname PY@tok@kp\endcsname{\def\PY@tc##1{\textcolor[rgb]{0.00,0.50,0.00}{##1}}}
\expandafter\def\csname PY@tok@kt\endcsname{\def\PY@tc##1{\textcolor[rgb]{0.69,0.00,0.25}{##1}}}
\expandafter\def\csname PY@tok@o\endcsname{\def\PY@tc##1{\textcolor[rgb]{0.40,0.40,0.40}{##1}}}
\expandafter\def\csname PY@tok@ow\endcsname{\let\PY@bf=\textbf\def\PY@tc##1{\textcolor[rgb]{0.67,0.13,1.00}{##1}}}
\expandafter\def\csname PY@tok@nb\endcsname{\def\PY@tc##1{\textcolor[rgb]{0.00,0.50,0.00}{##1}}}
\expandafter\def\csname PY@tok@nf\endcsname{\def\PY@tc##1{\textcolor[rgb]{0.00,0.00,1.00}{##1}}}
\expandafter\def\csname PY@tok@nc\endcsname{\let\PY@bf=\textbf\def\PY@tc##1{\textcolor[rgb]{0.00,0.00,1.00}{##1}}}
\expandafter\def\csname PY@tok@nn\endcsname{\let\PY@bf=\textbf\def\PY@tc##1{\textcolor[rgb]{0.00,0.00,1.00}{##1}}}
\expandafter\def\csname PY@tok@ne\endcsname{\let\PY@bf=\textbf\def\PY@tc##1{\textcolor[rgb]{0.82,0.25,0.23}{##1}}}
\expandafter\def\csname PY@tok@nv\endcsname{\def\PY@tc##1{\textcolor[rgb]{0.10,0.09,0.49}{##1}}}
\expandafter\def\csname PY@tok@no\endcsname{\def\PY@tc##1{\textcolor[rgb]{0.53,0.00,0.00}{##1}}}
\expandafter\def\csname PY@tok@nl\endcsname{\def\PY@tc##1{\textcolor[rgb]{0.63,0.63,0.00}{##1}}}
\expandafter\def\csname PY@tok@ni\endcsname{\let\PY@bf=\textbf\def\PY@tc##1{\textcolor[rgb]{0.60,0.60,0.60}{##1}}}
\expandafter\def\csname PY@tok@na\endcsname{\def\PY@tc##1{\textcolor[rgb]{0.49,0.56,0.16}{##1}}}
\expandafter\def\csname PY@tok@nt\endcsname{\let\PY@bf=\textbf\def\PY@tc##1{\textcolor[rgb]{0.00,0.50,0.00}{##1}}}
\expandafter\def\csname PY@tok@nd\endcsname{\def\PY@tc##1{\textcolor[rgb]{0.67,0.13,1.00}{##1}}}
\expandafter\def\csname PY@tok@s\endcsname{\def\PY@tc##1{\textcolor[rgb]{0.73,0.13,0.13}{##1}}}
\expandafter\def\csname PY@tok@sd\endcsname{\let\PY@it=\textit\def\PY@tc##1{\textcolor[rgb]{0.73,0.13,0.13}{##1}}}
\expandafter\def\csname PY@tok@si\endcsname{\let\PY@bf=\textbf\def\PY@tc##1{\textcolor[rgb]{0.73,0.40,0.53}{##1}}}
\expandafter\def\csname PY@tok@se\endcsname{\let\PY@bf=\textbf\def\PY@tc##1{\textcolor[rgb]{0.73,0.40,0.13}{##1}}}
\expandafter\def\csname PY@tok@sr\endcsname{\def\PY@tc##1{\textcolor[rgb]{0.73,0.40,0.53}{##1}}}
\expandafter\def\csname PY@tok@ss\endcsname{\def\PY@tc##1{\textcolor[rgb]{0.10,0.09,0.49}{##1}}}
\expandafter\def\csname PY@tok@sx\endcsname{\def\PY@tc##1{\textcolor[rgb]{0.00,0.50,0.00}{##1}}}
\expandafter\def\csname PY@tok@m\endcsname{\def\PY@tc##1{\textcolor[rgb]{0.40,0.40,0.40}{##1}}}
\expandafter\def\csname PY@tok@gh\endcsname{\let\PY@bf=\textbf\def\PY@tc##1{\textcolor[rgb]{0.00,0.00,0.50}{##1}}}
\expandafter\def\csname PY@tok@gu\endcsname{\let\PY@bf=\textbf\def\PY@tc##1{\textcolor[rgb]{0.50,0.00,0.50}{##1}}}
\expandafter\def\csname PY@tok@gd\endcsname{\def\PY@tc##1{\textcolor[rgb]{0.63,0.00,0.00}{##1}}}
\expandafter\def\csname PY@tok@gi\endcsname{\def\PY@tc##1{\textcolor[rgb]{0.00,0.63,0.00}{##1}}}
\expandafter\def\csname PY@tok@gr\endcsname{\def\PY@tc##1{\textcolor[rgb]{1.00,0.00,0.00}{##1}}}
\expandafter\def\csname PY@tok@ge\endcsname{\let\PY@it=\textit}
\expandafter\def\csname PY@tok@gs\endcsname{\let\PY@bf=\textbf}
\expandafter\def\csname PY@tok@gp\endcsname{\let\PY@bf=\textbf\def\PY@tc##1{\textcolor[rgb]{0.00,0.00,0.50}{##1}}}
\expandafter\def\csname PY@tok@go\endcsname{\def\PY@tc##1{\textcolor[rgb]{0.53,0.53,0.53}{##1}}}
\expandafter\def\csname PY@tok@gt\endcsname{\def\PY@tc##1{\textcolor[rgb]{0.00,0.27,0.87}{##1}}}
\expandafter\def\csname PY@tok@err\endcsname{\def\PY@bc##1{\setlength{\fboxsep}{0pt}\fcolorbox[rgb]{1.00,0.00,0.00}{1,1,1}{\strut ##1}}}
\expandafter\def\csname PY@tok@kc\endcsname{\let\PY@bf=\textbf\def\PY@tc##1{\textcolor[rgb]{0.00,0.50,0.00}{##1}}}
\expandafter\def\csname PY@tok@kd\endcsname{\let\PY@bf=\textbf\def\PY@tc##1{\textcolor[rgb]{0.00,0.50,0.00}{##1}}}
\expandafter\def\csname PY@tok@kn\endcsname{\let\PY@bf=\textbf\def\PY@tc##1{\textcolor[rgb]{0.00,0.50,0.00}{##1}}}
\expandafter\def\csname PY@tok@kr\endcsname{\let\PY@bf=\textbf\def\PY@tc##1{\textcolor[rgb]{0.00,0.50,0.00}{##1}}}
\expandafter\def\csname PY@tok@bp\endcsname{\def\PY@tc##1{\textcolor[rgb]{0.00,0.50,0.00}{##1}}}
\expandafter\def\csname PY@tok@fm\endcsname{\def\PY@tc##1{\textcolor[rgb]{0.00,0.00,1.00}{##1}}}
\expandafter\def\csname PY@tok@vc\endcsname{\def\PY@tc##1{\textcolor[rgb]{0.10,0.09,0.49}{##1}}}
\expandafter\def\csname PY@tok@vg\endcsname{\def\PY@tc##1{\textcolor[rgb]{0.10,0.09,0.49}{##1}}}
\expandafter\def\csname PY@tok@vi\endcsname{\def\PY@tc##1{\textcolor[rgb]{0.10,0.09,0.49}{##1}}}
\expandafter\def\csname PY@tok@vm\endcsname{\def\PY@tc##1{\textcolor[rgb]{0.10,0.09,0.49}{##1}}}
\expandafter\def\csname PY@tok@sa\endcsname{\def\PY@tc##1{\textcolor[rgb]{0.73,0.13,0.13}{##1}}}
\expandafter\def\csname PY@tok@sb\endcsname{\def\PY@tc##1{\textcolor[rgb]{0.73,0.13,0.13}{##1}}}
\expandafter\def\csname PY@tok@sc\endcsname{\def\PY@tc##1{\textcolor[rgb]{0.73,0.13,0.13}{##1}}}
\expandafter\def\csname PY@tok@dl\endcsname{\def\PY@tc##1{\textcolor[rgb]{0.73,0.13,0.13}{##1}}}
\expandafter\def\csname PY@tok@s2\endcsname{\def\PY@tc##1{\textcolor[rgb]{0.73,0.13,0.13}{##1}}}
\expandafter\def\csname PY@tok@sh\endcsname{\def\PY@tc##1{\textcolor[rgb]{0.73,0.13,0.13}{##1}}}
\expandafter\def\csname PY@tok@s1\endcsname{\def\PY@tc##1{\textcolor[rgb]{0.73,0.13,0.13}{##1}}}
\expandafter\def\csname PY@tok@mb\endcsname{\def\PY@tc##1{\textcolor[rgb]{0.40,0.40,0.40}{##1}}}
\expandafter\def\csname PY@tok@mf\endcsname{\def\PY@tc##1{\textcolor[rgb]{0.40,0.40,0.40}{##1}}}
\expandafter\def\csname PY@tok@mh\endcsname{\def\PY@tc##1{\textcolor[rgb]{0.40,0.40,0.40}{##1}}}
\expandafter\def\csname PY@tok@mi\endcsname{\def\PY@tc##1{\textcolor[rgb]{0.40,0.40,0.40}{##1}}}
\expandafter\def\csname PY@tok@il\endcsname{\def\PY@tc##1{\textcolor[rgb]{0.40,0.40,0.40}{##1}}}
\expandafter\def\csname PY@tok@mo\endcsname{\def\PY@tc##1{\textcolor[rgb]{0.40,0.40,0.40}{##1}}}
\expandafter\def\csname PY@tok@ch\endcsname{\let\PY@it=\textit\def\PY@tc##1{\textcolor[rgb]{0.25,0.50,0.50}{##1}}}
\expandafter\def\csname PY@tok@cm\endcsname{\let\PY@it=\textit\def\PY@tc##1{\textcolor[rgb]{0.25,0.50,0.50}{##1}}}
\expandafter\def\csname PY@tok@cpf\endcsname{\let\PY@it=\textit\def\PY@tc##1{\textcolor[rgb]{0.25,0.50,0.50}{##1}}}
\expandafter\def\csname PY@tok@c1\endcsname{\let\PY@it=\textit\def\PY@tc##1{\textcolor[rgb]{0.25,0.50,0.50}{##1}}}
\expandafter\def\csname PY@tok@cs\endcsname{\let\PY@it=\textit\def\PY@tc##1{\textcolor[rgb]{0.25,0.50,0.50}{##1}}}

\def\PYZbs{\char`\\}
\def\PYZus{\char`\_}
\def\PYZob{\char`\{}
\def\PYZcb{\char`\}}
\def\PYZca{\char`\^}
\def\PYZam{\char`\&}
\def\PYZlt{\char`\<}
\def\PYZgt{\char`\>}
\def\PYZsh{\char`\#}
\def\PYZpc{\char`\%}
\def\PYZdl{\char`\$}
\def\PYZhy{\char`\-}
\def\PYZsq{\char`\'}
\def\PYZdq{\char`\"}
\def\PYZti{\char`\~}
% for compatibility with earlier versions
\def\PYZat{@}
\def\PYZlb{[}
\def\PYZrb{]}
\makeatother



    
    % Prevent overflowing lines due to hard-to-break entities
    \sloppy
    % Setup hyperref package
    \hypersetup{
    breaklinks=true, % so long urls are correctly broken across lines
	pdftitle={\@title},
	pdfauthor={\@author},
	colorlinks=true, % color links 
	breaklinks=true, % enable to break long links
	urlcolor= blue,  % color for external links
	linkcolor= blue, % color for external links
	citecolor=blue,  
	bookmarksopen=false,
	pdftoolbar=false,
	pdfmenubar=false,
%      hidelinks
      }
    % Slightly bigger margins than the latex defaults
    \geometry{verbose,tmargin=1in,bmargin=1in,lmargin=1in,rmargin=1in}
    %listings configuration

\definecolor{mygreen}{rgb}{0,0.6,0}
\definecolor{mygray}{rgb}{0.5,0.5,0.5}
\definecolor{mymauve}{rgb}{0.58,0,0.82}
	\lstset{
language=Python,
commentstyle=\color{mygreen},
keywordstyle=\color{blue},
stringstyle=\color{mymauve},
xleftmargin= 1cm,
xrightmargin= 1cm,
showstringspaces=false,
	   breaklines=true,
	   texcl=false,
%	   basicstyle=\ttfamily,
frame=single,
frameround=tttt,
framesep=10pt,
%framexleftmargin=10pt,
%framexrightmargin =10pt,
%frameshape={RYRYNYYYY}{yny}{yny}{RYRYNYYYY} 
        inputencoding=utf8,
        extendedchars=true,
        literate=%
        {é}{{\'{e}}}1
        {è}{{\`{e}}}1
        {ê}{{\^{e}}}1
        {ë}{{\¨{e}}}1
        {É}{{\'{E}}}1
        {Ê}{{\^{E}}}1
        {û}{{\^{u}}}1
        {ù}{{\`{u}}}1
        {à}{{\`{a}}}1
        {ç}{{\c{c}}}1
        {Ç}{{\c{C}}}1
        {î}{{\^{i}}}1
        {Î}{{\^{I}}}1
}


%\usepackage{foo}

    \begin{document}
    
    
    \maketitle
    
\tableofcontents

    


    
    \hypertarget{wing-modeling}{%
\section{Wing modeling}\label{wing-modeling}}

    \hypertarget{naca-airfoils}{%
\subsection{NACA Airfoils}\label{naca-airfoils}}

    \hypertarget{symmetrical-4-digit-naca-foil}{%
\subsubsection{Symmetrical 4-digit NACA
foil}\label{symmetrical-4-digit-naca-foil}}

The formula for the shape of a NACA 00xx foil, with ``x'' being replaced
by the percentage of thickness to chord, is:
\begin{equation}\label{eq:series4sym_yt}
    \begin{aligned}
        y_{t}=5t\,\left[0.2969{\sqrt {x}}-0.1260x-0.3516x^{2}+0.2843x^{3}-0.1015x^{4}\right]
    \end{aligned}
\end{equation} * \(x\) is the position along the chord from 0 to 1.00,
(0 to 100%)
* \(y_{t}\) is the half thickness at a given value of
x (centerline to surface), and * \(t\) is the maximum thickness as a
fraction of the chord (so t gives the last two digits in the NACA
4-digit denomination divided by 100).
%
\begin{lstlisting}
#NACA 4-digit airfoil generator function
from jupyter_code_template import *
def gen_NACA(series = [6, 4, 12], N = 1000):
    """
        series is a list [a,b,x]
        -a = 100*m is the maximum camber,
        -b = 100*p is the location of maximum camber,
        -t = x/100 is the maximum thickness as a fraction of the chord
        -N is the number of x coordinates in total
        
        returns: N, x, y_c , y_U, y_L
        -N is the length of X
        -x is the x coordinate array
        -y_c is the mean camber line
        -y_U is the upper edge coorndinate array
        -y_L is the lower edge coordinate array
        
        Notes:
        -If a = 0 OR b = 0, then the foil is assumed to be symmertical 
        - in this case y_c is set to None
    """
    x = np.linspace(0, 1, N, endpoint=True)
    m = series[0]/100
    p = series[1]/10
    t = series[2]/100
    c = 1

    y_t = 5*t*(0.2969*sqrt(x)-0.1260*x-0.3516*x**2+0.2844*x**3-0.1015*x**4)
    
    if(m == 0 or p == 0):
        print("symmetric")
        return N, x, None, y_t, -1*y_t
    else:
        print("cambered")
        y_c = np.zeros(N)
        y_c[0:int(p*N)] = (m/p**2)*(2*p*(x[0:int(p*N)]/c)-(x[0:int(p*N)]/c)**2)
        y_c[int(p*N):] = (m/(1-p)**2)*((1-2*p)+2*p*(x[int(p*N):]/c)-(x[int(p*N):]/c)**2)

        dy_c = np.zeros(N)
        dy_c[0:int(p*N)] = (m/p**2)*(p-(x[0:int(p*N)]/c))
        dy_c[int(p*N):] = (m/(1-p)**2)*(p-(x[int(p*N):]/c))

        theta = np.arctan(dy_c)

        # x_U = x-y_t*sin(theta)
        # x_L= x+y_t*sin(theta)
        y_U = y_c+y_t*cos(theta)
        y_L = y_c-y_t*cos(theta)
    
        return N, x, y_c, y_U, y_L

def cambered_foil_plt(fig, ax, N, x, y_c, y_U, y_L, NACA):
    NACA_str = ''.join(map(str, NACA))
    ax = config_axis(ax, x_lim = (0, 1), Eng = False)

    ax.plot(x, y_c, label =  "Mean camber line: "+r"$y_c$")
    ax.plot(x, y_U, label = "Upper edge: " + r"$y_U$")
    ax.plot(x, y_L, label = "Lower edge: " + r"$y_L$")
    ax.fill_between(x, y_U, y_L, facecolor = "grey", alpha = 0.4, label = "Foil section")

    ax.legend(loc = "upper left", bbox_to_anchor = (1, 1))
    figcaption("Cambered 4-digit NACA " + NACA_str +" airfoil", label="fig:series4cam")
    plt.show()
    
def symmetric_foil_plt(fig, ax, N, x, y_c, y_U, y_L, NACA):
    NACA_str = ''.join(map(str, NACA))
    
    ax = config_axis(ax, x_lim = (0, 1), Eng = False)
    ax.plot(x, y_U, label = "Upper edge: " + r"$y_U$")
    ax.plot(x, y_L, label = "Lower edge: " + r"$y_L$")
    ax.fill_between(x, y_U, y_L, facecolor = "grey", alpha = 0.4, label = "Foil section")
    ax.legend(loc = "upper left", bbox_to_anchor = (1, 1))
    figcaption("Symmetric 4-digit NACA " + NACA_str +" airfoil", label="fig:series4sym")
    plt.show()

def xfoil_foilgen(series = [6, 4, 12], N = 100):
    """
        Takes in a 4-digit NACA code, and number of data points,
        it then generates the corresponding x and y cooridinates
        for use in XFOIL
    """
    N, x_i, y_c, y_U_i, y_L_i = gen_NACA(series, N = N)
    
    x_U = x_i[::-1]
    x_L = x_i
    x = np.concatenate((x_U, x_L))
    
    y_U = y_U_i[::-1]
    y_L = y_L_i
    y = np.concatenate((y_U, y_L))
    N = len(x)
    return N, x, y
\end{lstlisting}%
\begin{lstlisting}
from jupyter_code_template import *
NACA = [0, 0, 12]
NACA_str = ''.join(map(str, NACA))
N = 1000
N, x, y_c, y_U, y_L = gen_NACA(series=NACA, N = N)

fig1, ax1 = plt.subplots(1, 1, figsize = (12, 3))
symmetric_foil_plt(fig1, ax1, N, x, y_c, y_U, y_L, NACA)
\end{lstlisting}%
    \begin{Verbatim}[commandchars=\\\{\}]
symmetric

    \end{Verbatim}

    
    

    
    
    
    
\begin{figure}
    \begin{center}
        \adjustimage{max size={0.7\linewidth}{0.3\paperheight}}{UAV_Design_files/UAV_Design_4_3.png}
    \end{center}
    \caption{Symmetric 4-digit NACA 0012 airfoil}      
    \label{fig:series4sym}
\end{figure}
    
    \hypertarget{cambered-4-digit-naca-foil}{%
\subsubsection{Cambered 4-digit NACA
foil}\label{cambered-4-digit-naca-foil}}

The simplest asymmetric foils are the NACA 4-digit series foils, which
use the same formula as that used to generate the 00xx symmetric foils,
but with the line of mean camber bent. The formula used to calculate the
mean camber line is:

\begin{equation}\label{eq:series4cambered_yc}
    \begin{aligned}
        y_{c} =
        \begin{cases}
            \frac{m}{p^2}\left( 2p\left(\frac{x}{c}\right)-\left(\frac{x}{c}\right)^2 \right), & 0\leq x \leq pc\\
            \frac{m}{(1-p)^2}\left( (1-2p)+2p\left( \frac{x}{c}\right)-\left(\frac{x}{c}\right)^2 \right), & pc \leq x \leq c\\
        \end{cases}
    \end{aligned}
\end{equation} * \(m\) is the maximum camber (\(100 m\) is the first of
the four digits), * \(p\) is the location of maximum camber (\(10 p\) is
the second digit in the NACA xxxx description).

For this cambered airfoil, because the thickness needs to be applied
perpendicular to the camber line, the coordinates \((x_U,y_U)\) and
\((x_L,y_L)\), of respectively the upper and lower airfoil surface,
become: \begin{align}
    x_U &= x-y_t \sin(\theta), & y_U &= y_c +y_t\cos(\theta)\\
    x_L &= x+y_t \sin(\theta), & y_L &= y_c -y_t\cos(\theta)\\
\end{align}, where \begin{equation}
    \theta = \arctan\left(\frac{dy_c}{dx}\right)\\
\end{equation} \begin{equation}
    \begin{aligned}
        \frac{dy_c}{dx} =
        \begin{cases}
            \frac{2m}{p^2} \left( p-\frac{x}{c}\right), & 0\leq x \leq pc\\
            \frac{m}{(1-p)^2}\left( p-\frac{x}{c}\right), & pc \leq x \leq c\\
        \end{cases}
    \end{aligned}
\end{equation}
%
\begin{lstlisting}
from jupyter_code_template import *
NACA = [4, 4, 12]
N = 1000

N, x, y_c, y_U, y_L = gen_NACA(series=NACA, N = N)

fig2, ax2 = plt.subplots(1, 1, figsize = (12, 3))

cambered_foil_plt(fig2, ax2, N, x, y_c, y_U, y_L, NACA)
\end{lstlisting}%
    
    

    
    
    
    
\begin{figure}
    \begin{center}
        \adjustimage{max size={0.7\linewidth}{0.3\paperheight}}{UAV_Design_files/UAV_Design_6_2.png}
    \end{center}
    \caption{Cambered 4-digit NACA 4412 airfoil}      
    \label{fig:series4cam}
\end{figure}
    %
\begin{lstlisting}
from jupyter_code_template import *
NACA = [4.5, 4, 12]
N = 1000

N, x, y_c, y_U, y_L = gen_NACA(series=NACA, N = N)

fig2, ax2 = plt.subplots(1, 1, figsize = (12, 3))

cambered_foil_plt(fig2, ax2, N, x, y_c, y_U, y_L, NACA)
\end{lstlisting}%
    \begin{Verbatim}[commandchars=\\\{\}]
cambered

    \end{Verbatim}

    
    

    
    
    
    
\begin{figure}
    \begin{center}
        \adjustimage{max size={0.7\linewidth}{0.3\paperheight}}{UAV_Design_files/UAV_Design_7_3.png}
    \end{center}
    \caption{Cambered 4-digit NACA 4.5412 airfoil}      
    \label{fig:series4cam}
\end{figure}
    %
\begin{lstlisting}
N, x, y = xfoil_foilgen(series = [0, 0, 12], N = 160)
fig, ax = plt.subplots(1, 1, figsize = (12, 3))
ax = config_axis(ax, Eng = False)

x1 = x[int(N/2):]
y_U = y[:int(N/2)]
y_U = y_U[::-1]
y_L = y[int(N/2):]

ax.plot(x1, y_U, label = "Upper edge: " + r"$y_U$")
ax.plot(x1, y_L, label = "Lower edge: " + r"$y_L$")
ax.fill_between(x1, y_U, y_L, facecolor = "grey", alpha = 0.4, label = "Foil section")
ax.legend(loc = "upper left", bbox_to_anchor = (1, 1))
plt.show()
\end{lstlisting}%
    \begin{Verbatim}[commandchars=\\\{\}]
symmetric

    \end{Verbatim}

    
\begin{figure}
    \begin{center}
        \adjustimage{max size={0.7\linewidth}{0.3\paperheight}}{UAV_Design_files/UAV_Design_8_1.png}
    \end{center}
\end{figure}
    
    \hypertarget{digit-naca}{%
\subsubsection{5-digit NACA}\label{digit-naca}}

    \hypertarget{aeropy}{%
\subsection{AeroPy}\label{aeropy}}
%
\begin{lstlisting}
import aeropy.xfoil_module as xf
# xf.find_coefficients('naca0015', 5)
NACA = [4, 4, 12]
NACA_str = 'naca'+''.join(map(str, NACA))
# N, x, y_c, y_U, y_L = gen_NACA(NACA)
# filename = xf.file_name(NACA_str,  alfas=5, output = 'Polar')
# print(filename)
# xf.create_input(x, y_U, y_L, filename, True)
print(xf.find_coefficients('24112',5,10e4))
# xf.call(NACA_str, 5, 'Coordinates',  10e4, 0, False)
\end{lstlisting}%
    \begin{Verbatim}[commandchars=\\\{\}]

        ---------------------------------------------------------------------------

        FileNotFoundError                         Traceback (most recent call last)

        <ipython-input-3-b671d600c635> in <module>
          7 \# print(filename)
          8 \# xf.create\_input(x, y\_U, y\_L, filename, True)
    ----> 9 print(xf.find\_coefficients('24112',5,10e4))
         10 \# xf.call(NACA\_str, 5, 'Coordinates',  10e4, 0, False)
    

        C:\textbackslash{}Users\textbackslash{}Purco\textbackslash{}Anaconda3\textbackslash{}lib\textbackslash{}site-packages\textbackslash{}aeropy\textbackslash{}xfoil\_module.py in find\_coefficients(airfoil, alpha, Reynolds, iteration, NACA, delete, PANE, GDES)
        950     coefficients = \{\}
        951     \# Data from file
    --> 952     Data = output\_reader(filename, output='Polar', delete=False)
        953     for key in Data:
        954         try:
    

        C:\textbackslash{}Users\textbackslash{}Purco\textbackslash{}Anaconda3\textbackslash{}lib\textbackslash{}site-packages\textbackslash{}aeropy\textbackslash{}xfoil\_module.py in output\_reader(filename, separator, output, rows\_to\_skip, header, delete, structure, type\_structure)
        665         separator\_list = separator
        666     structure\_count = 0
    --> 667     with open(filename, "r") as myfile:
        668         \# Jump first lines which are useless
        669         for line in myfile:
    

        FileNotFoundError: [Errno 2] No such file or directory: 'Polar\_24112\_0500\_0500'

    \end{Verbatim}

    \hypertarget{uav-design-ideas}{%
\subsection{UAV design ideas}\label{uav-design-ideas}}

\begin{itemize}
\tightlist
\item
  Foam structure
\item
  Skin
\item
  Mould for rapid deployment

  \begin{itemize}
  \tightlist
  \item
    JIG for foam cutting instead?
  \end{itemize}
\item
  Pixhawk brain

  \begin{itemize}
  \tightlist
  \item
    Definitely --\textgreater{} I shoud probably source one
  \end{itemize}
\item
  AI autopilot

  \begin{itemize}
  \tightlist
  \item
    Long term goal, integrate with pixahawk
  \end{itemize}
\item
  Predefined wiring harness

  \begin{itemize}
  \tightlist
  \item
    Obviously
  \end{itemize}
\item
  Parachute for landing

  \begin{itemize}
  \tightlist
  \item
    Yes
  \end{itemize}
\item
  Sled takeoff

  \begin{itemize}
  \tightlist
  \item
    Yes
  \item
    Compressed air?

    \begin{itemize}
    \tightlist
    \item
      Not right now.
    \end{itemize}
  \item
    Elastic bands seem better.
  \item
    VTOL sled? Seems cool, why not?
  \end{itemize}
\item
  Design challenge:

  \begin{itemize}
  \tightlist
  \item
    Design for scaling
  \item
    Rapid prototyping
  \item
    Rapid deployment
  \item
    Minimal moving parts
  \end{itemize}
\item
  Pitch control
\item
  Torque vectoring for pitch control
\item
  Tail surface?
\item
  Linear motor over centre of gravity
\item
  Up-cambered wings
\item
  Reaper and Predator drones really caught my eye. These were basically
  just glider wings, set back on the fuselage to counter the weight of
  the camera equipment and sensors in the nose. The anhedral tail
  surfaces on the Predator are intended to assist in the prevention of
  prop-strike during take-off and landing -- plus they look cool.
\end{itemize}

    \begin{itemize}
\tightlist
\item
  Air time: 10\textasciitilde{}min
\item
  SDR \& GPU-Jetson

  \begin{itemize}
  \tightlist
  \item
    15 W to 20 W (achievable)
  \end{itemize}
\item
  GPS
\item
  Mini-airship
\item
  Does it need to move a distance

  \begin{itemize}
  \tightlist
  \item
    Tethering is difficult
  \item
    Yes we want it to move
  \end{itemize}
\item
  Low transmission power in RF, 100 mW cap (to be revised)
\item
  Try to get better power consumption estimates

  \begin{itemize}
  \tightlist
  \item
    best GPU?
  \item
    radar
  \item
    SDR
  \end{itemize}
\end{itemize}

    \hypertarget{jamming-system-uav-requirements}{%
\section{Jamming System UAV
Requirements}\label{jamming-system-uav-requirements}}

\textbf{2019/02/28}

\begin{itemize}
\tightlist
\item
  SDR Type: Nuand BladeRF 2.0
\item
  SDR Dimensions: 12 cm (17 cm with amplifiers) x 8 cm x 2.5 cm
\item
  SDR Power:(5 V ; 1 A) \(\implies\) 5 .0 W
\item
  SDR External Amplifier Power:

  \begin{itemize}
  \tightlist
  \item
    2 x PA: (2 x 0.3 W) \(\implies\) 0.6 W
  \item
    2 x LNA: (2 x 0.45 W) \(\implies\) 0.9 W
  \end{itemize}
\item
  Jammer Wingspan Requirements: 2.5 m
\item
  SDR Mass (with amplifiers and cover): 350 g
\item
  Odroid Processor Power \(\implies\) 20 W
\item
  Odroid Dimensions: 7 cm x 9 cm x 3 cm
\item
  Odroid Weight: 100 g
\item
  RG-58 Coaxial Cable: 37 g / m
\end{itemize}

    \hypertarget{rietfontein-anti-poaching}{%
\subsection{Rietfontein Anti-poaching}\label{rietfontein-anti-poaching}}

    \begin{itemize}
\tightlist
\item
  Rietfontein Anti-poaching

  \begin{itemize}
  \tightlist
  \item
    Need a drone.
  \item
    We can test at the Rietfontein
  \item
    We need to get specs
  \item
    Circumference of 30 km
  \item
    Diameter of 8 km
  \end{itemize}
\end{itemize}

    \hypertarget{design-requirements}{%
\subsection{Design requirements}\label{design-requirements}}

\begin{itemize}
\tightlist
\item
  UAV wing tips
\item
  Reflexed Wing tips
\item
  Fibre glass manufacturing don't so small jobs
\item
  They also don't
\end{itemize}

    \hypertarget{build-techniques}{%
\subsection{Build Techniques}\label{build-techniques}}

\begin{itemize}
\tightlist
\item
  Balsa

  \begin{itemize}
  \tightlist
  \item
    Hard to build
  \item
    Fragile
  \item
    Hard to repair
  \end{itemize}
\item
  Folded foam board

  \begin{itemize}
  \tightlist
  \item
    Quick cheap and easy
  \item
    Not much in the way of accuracy
  \end{itemize}
\item
  Coragated platic board

  \begin{itemize}
  \tightlist
  \item
    indestrutable
  \item
    inaccurate
  \end{itemize}
\item
  Foam board

  \begin{itemize}
  \tightlist
  \item
    Hot wire cutter
  \item
    really fragile

    \begin{itemize}
    \tightlist
    \item
      packing tape
    \item
      laminate sheet
    \item
      fibre glass onto foam (promising)
    \end{itemize}
  \end{itemize}
\end{itemize}

    \hypertarget{tutorials}{%
\subsection{Tutorials}\label{tutorials}}

\begin{itemize}
\tightlist
\item
  https://www.instructables.com/id/Black-Eagle-aerial-mapping-UAV/
\item
  https://www.youtube.com/watch?v=Xafg-o83L94
\item
  https://www.robotshop.com/community/tutorials/show/how-to-make-a-drone-uav-lesson-8-airplanes
\end{itemize}

    \hypertarget{genetic-algorithm-implementation}{%
\subsection{Genetic Algorithm
implementation}\label{genetic-algorithm-implementation}}

    \hypertarget{basic-ga-tutorial}{%
\subsubsection{Basic GA tutorial}\label{basic-ga-tutorial}}

\begin{itemize}
\tightlist
\item
  https://towardsdatascience.com/genetic-algorithm-implementation-in-python-5ab67bb124a6
  \cite{gad_genetic_2018}
\end{itemize}
%
\begin{lstlisting}
def cal_pop_fitness(equation_inputs, pop):
     # Calculating the fitness value of each solution in the current population.
     # The fitness function calculates the sum of products between each input
     # and its corresponding weight.
     fitness = np.sum(pop*equation_inputs, axis=1)
     return fitness

def select_mating_pool(pop, fitness, num_parents):

    # Selecting the best individuals in the current generation as parents
    # for producing the offspring of the next generation.

    #generate empty parent population array
    parents = np.empty((num_parents, pop.shape[1]))

    #iterate through the number of parents
    for parent_num in range(num_parents):
        #find which memper had max fitness
        max_fitness_idx = np.where(fitness == np.max(fitness))
        max_fitness_idx = max_fitness_idx[0][0]

        #choose the individual from the previous population with the maximum
        #fitness to be a parent for the next generation.
        parents[parent_num, :] = pop[max_fitness_idx, :]

        #set the current maximum fitness to be some small value so that on the
        #next iteration of the loop the next best individual will be selected
        #i.e. take this one out of the running
        fitness[max_fitness_idx] = -99999999999
    return parents

def crossover(parents, offspring_size):

    offspring = np.empty(offspring_size)
    # The point at which crossover takes place between two parents.
    #Usually, it is at the center.
    crossover_point = int(offspring_size[1]/2)

    for k in range(offspring_size[0]):
        # Index of the first parent to mate.
        parent1_idx = k%parents.shape[0]
        # Index of the second parent to mate.
        parent2_idx = (k+1)%parents.shape[0]
        # The new offspring will have its first half of its genes taken from the first parent.
        offspring[k, 0:crossover_point] = parents[parent1_idx, 0:crossover_point]
        # The new offspring will have its second half of its genes taken from the second parent.
        offspring[k, crossover_point:] = parents[parent2_idx, crossover_point:]
    return offspring

def mutation(offspring_crossover):

    # Mutation changes a single gene in each offspring randomly.

    for idx in range(offspring_crossover.shape[0]):

        # The random value to be added to the gene.

        random_value = np.random.uniform(-1.0, 1.0, 1)

        offspring_crossover[idx, 4] = offspring_crossover[idx, 4] + random_value

    return offspring_crossover
\end{lstlisting}%
\begin{lstlisting}
from jupyter_code_template import *

#Inputs to the function to be minimised
equation_inputs = [4,-2,3.5,5,-11,-4.7]

#parameters to be optimised
N_w = len(equation_inputs)

#the number of individuals in a population
N_ind = 8

#the population shape
pop_shape = (N_ind, N_w)

#seed the population
new_pop = np.random.uniform(-4.0, 4.0, size = pop_shape)

#the number of generations
N_gen = 10
new_parents_mating = 4
fit_arr = []
for gen in range(N_gen):
    fitness = cal_pop_fitness(equation_inputs, new_pop)
    parents = select_mating_pool(new_pop, fitness, new_parents_mating)
    offspring_crossover = crossover(parents, offspring_size = (pop_shape[0]-parents.shape[0], N_w))
    offspring_mutation = mutation(offspring_crossover)
    new_pop[0:parents.shape[0], :] = parents
    new_pop[parents.shape[0]:, :] = offspring_mutation
    fit = max(sum(new_pop*equation_inputs, axis=1))
    fit_arr = np.append(fit_arr, fit)

# Getting the best solution after iterating finishing all generations.
#At first, the fitness is calculated for each solution in the final generation.
fitness = cal_pop_fitness(equation_inputs, new_pop)
# Then return the index of that solution corresponding to the best fitness.
best_match_idx = np.where(fitness == max(fitness))

print("Best solution : ", new_pop[best_match_idx, :])
print("Best solution fitness : ", fitness[best_match_idx])

fig, ax = plt.subplots(1, 1, figsize = (12, 3))
ax = config_axis(ax, Eng = False)
ax.plot(fit_arr)
plt.show()
\end{lstlisting}%
    \begin{Verbatim}[commandchars=\\\{\}]
Best solution :  [[[ 3.17667176 -0.35600885  1.92396877  2.49634592 -6.41061891
   -2.64615371]]]
Best solution fitness :  [115.58805545]

    \end{Verbatim}

    
    \begin{verbatim}
<Figure size 3600x900 with 1 Axes>
    \end{verbatim}

    
    \hypertarget{naive-ga-with-aeropy}{%
\subsubsection{Naive GA with aeropy}\label{naive-ga-with-aeropy}}
%
\begin{lstlisting}
from jupyter_code_template import *
from numpy.random import randint
np.random.seed(20190321)

N_ind = 10
N_params = 3
pop = np.empty((N_ind, 3), int)
#seed population
for i in range(N_ind):
    pop[i, :] = [randint(0, 7, None, int), randint(0, 7, None, int), randint(10, 20, None, int)]

\end{lstlisting}%
    
    
    \begin{verbatim}
array([[ 6,  3, 19],
       [ 2,  2, 18],
       [ 4,  2, 16],
       [ 1,  0, 18],
       [ 2,  3, 15],
       [ 5,  1, 17],
       [ 0,  6, 15],
       [ 0,  1, 17],
       [ 3,  4, 15],
       [ 4,  6, 17]])
    \end{verbatim}

    

    \hypertarget{references}{%
\end{document}
